%LATEX Header
\documentclass[12pt]{report}
\usepackage[letterpaper]{geometry}
\geometry{top=1.0in, bottom=1.0in, left=1.0in, right=1.0in}
\usepackage{times}

% Title Page
\title{HIV/AIDS in South Africa \\ \normalsize{H312 Term Paper}}
\author{Torben Rasmussen}
\date{2/28/2011}

\begin{document}
\maketitle

%\section{Introduction}
HIV/AIDS is known as \emph{the} epidemic of the 21st century, and its impact is felt strongly on a global scale.
However, different parts of the world are affected on different levels.  
South Africa, a modern, developed country, has felt the impact of HIV/AIDS on an unfathomable level.
This disease has become a serious issue in South Africa because of the country's social and political climate, public perception of the disease, and infection vectors that are difficult to deal with.  
However, modern political policies, third party aid organizations, and education of the public can make a profound difference when it comes to fighting HIV/AIDS in South Africa.

%\section{Background}
South Africa is a parliamentary democracy composed of nine distinct provinces, located on the southern tip of the African continent.
The country is bordered to the north by Namibia, Botswana, and Zimbabwe, and to the east by Mozambique and Swaziland.  
The country of Lesotho is surrounded completely by South Africa.
The capital of South Africa is Johannesburg.
The country has 1,221,037 $km^2$ of land, and a population of almost 50 million.  
The ethnic groups in South Africa are 79.4\% black, 9.2\% white, 8.8\% ``coloured'' (mixed race), and 2.6\% Asian\cite{wiki-sa}.
The 2009 estimate for GDP per capita was \$10,243, this is approximately the global average in that year, and exceptionally high for a nation in sub-Saharan Africa\cite{gap-gdp}.

AIDS was first diagnosed in two patients in South Africa in 1983.
The first reported death caused by AIDS was caused in the same year.
The number of patients diagnosed with AIDS stepped up to 46 in 1986.
In 1990, it was estimated that 140,000 people were living with HIV.
By 1993, that number had increased to 610,000.
This number continued to increase at an alarming rate.  
In 1999, the figure had reached 3,800,000.
In 2003, when the rate finally began to slow, approximately 5.1 million people were living with HIV.
The number has increase at a less significant rate since then, and there are currently estimated to be about 5.6 million infected\cite{gap-prevalence}.

%\section{Prevalence and Infection}
The prevalence of HIV/AIDS in South Africa is the highest of any country in the world, with 17.80\% of the 49 million inhabitants diagnosed\cite{cia}.
However, the group most likely to be infected with HIV, those aged 15-20, has actually seen a decline in incidence rates from 2002 to 2008\cite{shisana}.
For example, 20-year olds went from a 2.2\% incidence rate in 2005, to a 1.7\% incidence rate in 2008.
The overall trend of HIV incidence shows a very high rate from 1992 to 1997, and an almost equal and opposite decline from 1998 to 2008\cite{gap-incidence}.

%\subsection{Co-infection of HIV and Tuberculosis}
While HIV/AIDS has caused much suffering in South Africa, tuberculosis is still the leading cause of death.
HIV/AIDS infections go hand-in-hand with many cases of tuberculosis.
A weakened immune system caused by HIV/AIDS increases the risk of a tuberculosis infection.
Also, tuberculosis can accelerate the progress of HIV.
South Africa has a tuberculosis/HIV co-infection rate of almost 75\% of those infected with HIV.
Even though the country only accounts for 0.7\% of the world's population, it accounts for 28\% of those co-infected with tuberculosis and HIV\cite{avert}.

%\subsection{Affected Demographics}
It is important when discussing HIV/AIDS in South Africa to talk about which demographics are the most at risk for both being infected with, and infecting others with the disease.
This is critical because it both gives insight into which demographics need aid and education, as well as clarifying which socio-economic situations bring about high transmission rates of HIV.

%\subsection{Migrant Workers}
One critical change that affected the spread of HIV/AIDS in South Africa was the end of apartheid in the 1980s\cite{wiki-apartheid}.
This drastic regime change brought a massive influx of migrant workers, and created chaos in terms of managing HIV infection vectors.
While it is not completely known to what extent the migrant worker population affected the spread of this disease, it is not hard to imagine that groups of highly mobile, possibly sexually promiscuous, possibly HIV positive men could be potentially devastating disease vectors\cite{migration}.

%\subsection{Children and Families}
South Africa's HIV/AIDS epidemic has had a profound negative impact on children in various ways.  
In 2009, there were approximately 330,000 people under 15 living with AIDS, which is twice the number as in 2001.
There are two main infection routes for HIV: heterosexual sex, and mother-to-child transmission.  The transmission of HIV from mother to child happens approximately 11\% of the time.
This will have a drastic effect on the child's health, as its family will most likely already be struggling with the disease\cite{avert}.

%\subsection{Orphaned Children}
There are an estimated 1.9 million orphaned South African children where one or more of the parents are deceased.
Approximately 50\% of these deaths are the direct result of HIV/AIDS\cite{avert}.
These orphaned children have to either be taken care of by their older relatives, or fend for themselves.
Some of the orphaned children are taken care of by institutions that are able to deliver necessary care on a relatively large scale.
This support does well for the young children affected by HIV, but improvements need to be made in order to promote care of those youth who survive into adulthood.

%\subsection{Gender Violence and Inequality}
Physical and sexual violence against women is disturbingly common in South Africa. 
More than 40\% of South African men claimed to have been physically abusive to an intimate partner.
Over 25\% of men reported having raped a woman, with 5\% having raped a woman in the past year\cite{avert}.
This apparent gender-inequality is a large factor in the country's HIV/AIDS epidemic.  
Men who rape women are less likely to wear a condom, so the transmission rate of the disease will be hard to control in the case of these individuals.
Also, in a culture where women do not feel empowered, their ability to advocate safe sex practices is greatly diminished.

%\subsection{Violence Against Victims}
Fear of violence plays a large role in whether a woman seeks HIV testing or treatment.  
They may be reluctant to discover their HIV positive status, as it could lead to additional physical violence directed at them.
To stop this cycle of violence, men must rethink their perceptions of masculinity, and how their actions can contribute to violence and, in turn, the transmission of HIV.
South Africa, as well as other sub-Saharan African nations, are working toward growing men's organizations to work for this type of change\cite{eldis}.

%\subsection{Virgin Myth}
One other key misconception about HIV/AIDS is the so-called ``Virgin Myth''.
The false idea is that HIV/AIDS can be cured through sex with a virgin.
This myth can be traced back to the Victorian times, where the claim was that sex with a virgin could cure other sexually transmitted diseases.
This dangerous idea has lead to the shocking rape of toddlers and infants throughout South Africa, as well as the rape of other vulnerable young women\cite{rape}.
Rural areas are particularly affected by this myth because those afflicted by AIDS are desperate to find a virgin to cure their ailment, and young girls seem like the most available resource.
This myth not only causes unforgivable violence against women and girls, but also increases the spread of HIV/AIDS throughout South Africa and beyond.

%\subsection{Prostitution}
An additional social group at risk of becoming infected or transmitting HIV/AIDS is sex workers.
HIV prevalence among female prostitutes can be many times higher than the general population.  
Prostitutes are very likely to become unknowingly infected with HIV\cite{avert}.
Once infected, the disease will most likely spread to a number of different people before it is detected.
Prostitution can not be eliminated, but efforts must be made to test them for HIV as well as educate them about safe sex practices.


%\section{Preventative Measures}
HIV/AIDS is unique in that there is no known cure, and treatments, while much improved in the past two decades, are relatively ineffective.  
This, combined with the long latent phase of the disease, requires that prevention and HIV/AIDS awareness be the main focus of healthcare efforts.

%\subsection{Aids Awareness}
South Africa currently has a number of large-scale communication-based campaigns that focus on awareness of HIV/AIDS, as well as more general health issues.
The HIV Counseling and Testing campaign began in April 2010, and was aimed at improving AIDS awareness.
The government aims to open up a dialogue about HIV/AIDS on a national level by publicising the availability of free testing and counseling in health clinics.
They aim to accomplish this through door-to-door campaigning, billboard messages, and using word-of-mouth to highlight peoples' experiences and to expel the myths and stigma of HIV.

%\subsection{Condom Use}
One of the most critical steps that any country has to take in fighting the AIDS epidemic is promoting the distribution and widespread use of condoms.
The usage of condoms in South Africa is growing.  The percentage of people using a condom during their last sexual encounter increased from 27\% in 2002, to 35\% in 2005, and finally, 62\% in 2008\cite{shisana}.

%\section{Political Efforts and Problems}
One of the most critical factors determining a country's success at fighting the AIDS epidemic is its leadership.  
Strong, early political measures will ensure that any threat of infection is greatly reduced, and that the people stay well informed of what they must do to stay healthy.
On the other hand, poor leadership and misinformation can have exactly the opposite effect on combating epidemics.
Unfortunately, the latter was the case for South Africa.
%\subsection{Denial}
Thabo Mbeki, the president of South Africa from 1999 to 2008 would seek advice of AIDS deniers when making policy decisions.  
He even went as far as to appoint a number of such people to his Presidential AIDS advisory panel.  
Both Mbeki and his health minister, Manto Tshabalala-Msimang questioned the effectiveness of antiretroviral therapies.
His health minister even went as far as to call for the consumption of beetroot and garlic as a way of fighting HIV infection.
Mbeki treated HIV and AIDS as separate issues, when he would acknowledge their existence at all.
Even after Mbeki gave up his views that AIDS was not caused by a virus, he still showed little or no support for anti-AIDS measures that were so critical for his country's well-being.

%\subsection{World Bank Warning}
In 2003, the World Bank stated that South Africa will suffer "a complete economic collapse" within four generations if it failed to act to combat the HIV/AIDS epidemic.  
The authors of this report emphasized that the disease had the potential to turn South Africa, considered middle class, into a poor African nation, similar to Kenya.
At this time, President Thabo Mbeki was distancing himself from his suggestion that HIV was not the cause of AIDS.  
Even so, many of his actions in the fight against HIV/AIDS were viewed thus far as weak.
The main reason AIDS had the potential to disrupt the country's economy was that it was most prevalent not in the elderly and very young, but in the young adult community; those most responsible for creating a thriving economy.
At this time, Nelson Mandela, the former South African president, called for increased financial support of AIDS programs from European countries and Japan.

%\section{External Aid}
Often, HIV/AIDS is a challenge that a government can not cope with on its own.  
The anti HIV/AIDS initiative in South Africa has received a large amount of support form a number of organizations, most notably the Clinton Foundation, the Bill and Melinda Gates Foundation, and the President's Emergency Plan for AIDS Relief.

%\subsection{Clinton Foundation}
In 2003 the Clinton Foundation's HIV/AIDS initiative, along with four generic drug companies, offered a deal to reduce the cost of AIDS treatments by more than one-third from already discounted prices.
This deal was made available to South Africa, as well as several other African and Caribbean countries.
The plan was formed with support from 25 advisors from the Clinton foundation in conjunction with the South African government.

The plan emphasized the importance of South Africa paying for the "biggest slice" of the program, as its government was reluctant to accept aid from other African nations, and wanted to ensure funding was in place for its AIDS patients.
Like has been stated, and as is true of a number of developing African nations, South Africa was skeptical of aid promises.
Health Minister Manto Tshabalala-Msimang claimed that this plan would not have been feasible had there not been a drastic reduction in the price of AIDS treatments over the years.


%\subsection{Gates Foundation}
In February 2004, the Bill and Melinda Gates foundation offered the Aeras Foundation a grant of \$82 million to develop and license an improved TB vaccine for use in high burden countries\cite{gates}.
While this grant does not target HIV/AIDS specifically, an effective tuberculosis vaccine will greatly improve the health of South Africans, and reduce the risk of HIV/TB co-infections.

%\subsection{PEPFAR}
Another external source of aid comes from the President's Emergency Plan for AIDS Relief (PEPFAR).
This is a United States sponsored aid initiative with the goal of fighting HIV/AIDS.
Some of the achievements during the 2010 fiscal year include: 
    917,700 individuals receiving antiretroviral treatment, 
    2.1 million HIV-positive individuals receiving care and support, 
    386,400 orphans and vulnerable children receiving support, 
    682,400 pregnant women receiving HIV treatments, 
    207,100 HIV+ pregnant women receiving antiretroviral prophylaxis for the prevention of mother-to-child transmission, 
    5 million individuals receiving testing and counseling, 
    and 39,000 estimated infant HIV infections averted\cite{pepfar}.

%\section{Conclusion}
The HIV/AIDS epidemic has taken a large toll on most of the world, and no place has been affected more than South Africa.
This country has experienced the consequences of poor leadership in the face of an epidemic, as well as the problems of the stigma and misconceptions surrounding the disease.
In the face of all of this, a number of groups were able to stand and make a difference in the lives of South Africans.
A single entity, however, will not make the difference in the fight against HIV/AIDS in South Africa and elsewhere.
The global fight against this disease requires a large, directed effort at educating at-risk people.

It is now, a decade into the 21st century, that we see a positive change in the outlook of South Africa's HIV/AIDS situation.
However, this is not the time to become complacent, as the youth of this generation are growing up with HIV/AIDS as the norm, and that makes the epidemic all the more serious.


\begin{thebibliography}{99}
    \bibitem{avert}``HIV \& ``AIDS in South Africa.'',
        AIDS \& HIV Information from the AIDS Charity AVERT. Web. 28 Feb. 2011. $<$http://www.avert.org/aidssouthafrica.htm$>$.
    \bibitem{shisana}``South African National HIV Prevalence, Incidence, Behaviour and Communication Survey, 2008''
        HSRC Press, 2009. Web. 28 Feb. 2011. $<$http://www.hsrc.ac.za/Document-3238.phtml$>$. 
    \bibitem{wiki-aids}``AIDS in South Africa'',
        Wikipedia. Web. 28 Feb. 2011. $<$http://en.wikipedia.org/wiki/AIDS\_in\_South\_Africa$>$.
    \bibitem{wiki-sa}``South Africa'',
        Wikipedia. Web. 28 Feb. 2011. $<$http://en.wikipedia.org/wiki/South\_africa$>$
    \bibitem{wiki-apartheid}``South Africa Under Apartheid''
        Wikipedia. Web. 28 Feb. 2011. $<$http://en.wikipedia.org/wiki/South\_Africa\_under\_apartheid$>$
    \bibitem{cia}``The World Factbook'',
        CIA. Web. 28 Feb. 2011. $<$https://www.cia.gov/library/publications/the-world-factbook/rankorder/2155rank.html$>$
    \bibitem{gap-prevalence}``AIDS Prevalence in South Africa'',
        Gapminder.org. Web. 28 Feb. 2011. $<$www.bit.ly/dMjEgl$>$
    \bibitem{gap-incidence}``AIDS Incidence in South Africa'',
        Gapminder.org. Web. 28 Feb. 2011. $<$www.bit.ly/ez5eGm$>$
    \bibitem{gap-gdp}``GDP per Capita of South Africa'',
        Gapminder.org. Web. 28 Feb. 2011. $<$www.bit.ly/etfGeP$>$
    \bibitem{gates}``Aeras Global TB Vaccine Foundation'',
        Bill and Melinda Gates Foundation. Web. 28 Feb. 2011 $<$http://www.gatesfoundation.org/grants-2004/Pages/Aeras-Global-TB-Vaccine-Foundation-OPP29789.aspx$>$
    \bibitem{pepfar}``Partnership to Fight HIV/AIDS in South Africa'',
        PEPFAR. Web. 28 Feb. 2011 $<$http://www.pepfar.gov/countries/southafrica/index.htm$>$
    \bibitem{migration}``Spaces Of Vulnerability: Migration And HIV/AIDS In South Africa'', 
        South African Migration Project. Web. 28 Feb. 2011 $<$http://www.queensu.ca/samp/sampresources/samppublications/policyseries/policy24.htm$>$
    \bibitem{rape}Richter, Linda M.,
        "Baby Rape in South Africa." Child Abuse Review 12.6 (2003): 392-400.
    \bibitem{world-bank}Dyer, Geoff.,
        "'Economic Collapse in S Africa Without Action on Aids' WORLD BANK WARNING." Financial Times [London] 15 July 2003, Edition 2 ed.: 11. Print.
    \bibitem{clinton}Schoofs, Mark.,
        "South Africa Reverses Course On AIDS Drugs." Wall Street Journal [New York] 20 Nov. 2003, Eastern Edition ed., sec. B: 1. Print.
    \bibitem{eldis}``The Links Between Violence Against Women and HIV and AIDS'',
        IDS Health \& Development Information Team. Web. 28 Feb. 2011. $<$http://www.eldis.org/UserFiles/File/VAW-1.pdf$>$
\end{thebibliography}
\end{document}
