%LATEX Header
%\documentclass[letterpaper,12pt]{article}
\documentclass[12pt]{report}
\usepackage[letterpaper]{geometry}
\geometry{top=1.0in, bottom=1.0in, left=1.0in, right=1.0in}
\usepackage{times}
\usepackage{enumerate}

%Fancy-header package to modify header/page numbering (insert last name)
\usepackage{fancyhdr}
\lhead{} 
\chead{} 
\rhead{Rasmussen \thepage} 
\lfoot{} 
\cfoot{} 
\rfoot{} 
\renewcommand{\headrulewidth}{0pt} 
\renewcommand{\footrulewidth}{0pt} 
%To make sure we actually have header 0.5in away from top edge
%12pt is one-sixth of an inch. Subtract this from 0.5in to get headsep value
\setlength\headsep{0.333in}
\setlength{\headheight}{15pt}

% Title Page
\title{HIV/AIDS in South Africa \\ \normalsize{H312 Term Paper}}
\author{Torben Rasmussen}
\date{2/28/2011}

\begin{document}
\pagestyle{fancy}
\maketitle
\newpage

\section{Introduction}
HIV/AIDS is known as \emph{the} epidemic of the 21st century, and its impact is felt strongly on a global scale.
However, different parts of the world are affected on different levels.  
South Africa, a modern, developed country, has felt the impact of HIV/AIDS in a huge way.  
This disease has become a serious issue in South Africa because of its social and political climate, public perception of the disease, and infection vectors that are difficult to deal with. 

\section{Background}
AIDS was first diagnosed in two patients in South Africa in 1983.  The first reported death caused by AIDS was caused in the same year.
The number of patients diagnosed with AIDS stepped up to 46 in 1986.

\section{Prevalence and Infection}
The prevalence of HIV/AIDS in South Africa is the highest of any country in the world, with 17.80\% of the 49 million inhabitants diagnosed\cite{cia}.
The group most likely to be infected with HIV, those aged 15-20, has actually seen a decline in incidence rates from 2002 to 2008\cite{shisana}.
For example, 20-year olds went from a 2.2\% incidence rate in 2005, to a 1.7\% incidence rate in 2008.

\subsection{Co-infection of HIV and Tuberculosis}
While HIV/AIDS has caused much suffering in South Africa, tuberculosis is still the leading cause of death.
Still, HIV/AIDS and tuberculosis go hand-in-hand.
A weakened immune system caused by HIV/AIDS increases the risk of a tuberculosis infection.
Also, tuberculosis can accelerate the progress of HIV.
South Africa has a prevalence of almost 75\% of those infected with HIV who also have tuberculosis.
Even though the country only accounts for 0.7\% of the world's population, it accounts for 28\% of those co-infected with tuberculosis and HIV.

\subsection{Affected Demographics}

\section{Social Impacts}

\subsection{Children and Families}
South Africa's HIV/AIDS epidemic has had a profound negative impact on children in various ways.  
In 2009, there were approximately 330,000 people under 15 living with AIDS, which is twice the number as in 2001.
There are two main infection routes for HIV: heterosexual sex, and mother-to-child transmission.  The transmission of HIV from mother to child happens approximately 11\% of the time.
This will have a drastic effect on the child's health, as its family will most likely already be struggling with the disease\cite{avert}.

\subsection{Aids Awareness}
South Africa currently has a number of large-scale communication-based campaigns that focus on awareness of HIV/AIDS, as well as more general health issues.
The HIV Counseling and Testing campaign began in April 2010, and was aimed at improving AIDS awareness.
The governmant aims to open up a dialogue about HIV/AIDS on a national level by publicising the availability of free testing and counseling in health clinics.
They aim to accomplish this through door-to-door campainging, billboard messages, and using word-of-mouth to highlight peoples' experiences and to expel the myths and stigma of HIV.


\subsection{Condom Use}
\subsection{Gender Violence and Inequality}
\subsection{Orphaned Children}

\section{Political Efforts and Problems}
\subsection{Denial}
Thabo Mbeki, the president of South Africa from 1999 to 2008 would seek advice of AIDS deniers when making policy decisions.  
He even went as far as to appoint a number of such people to his Presidential AIDS advisory panel.  
Both Mbeki and his health minister, Manto Tshabalala-Msimang questioned the effectiveness of antiretroviral therapies, and called for the consumption of beetroot and garlic as a way of fighting HIV infection.

\section{Conclusion}
In conclusion, stuff.
\begin{thebibliography}{99}
    \bibitem{avert}``HIV \& ``AIDS in South Africa.'',
        AIDS \& HIV Information from the AIDS Charity AVERT. Web. 28 Feb. 2011. $<$http://www.avert.org/aidssouthafrica.htm$>$.
    \bibitem{shisana}``South African National HIV Prevalence, Incidence, Behaviour and Communication Survey, 2008''
        HSRC Press, 2009. Web. 28 Feb. 2011. $<$http://www.hsrc.ac.za/Document-3238.phtml$>$. 
    %\bibitem[]{<+bibkey+>} <++>
    %    <++>
    \bibitem{aidswiki}http://en.wikipedia.org/wiki/AIDS\_in\_South\_Africa
    \bibitem{sawiki}http://en.wikipedia.org/wiki/South\_africa
    \bibitem{doh}http://www.doh.gov.za/docs/reports/2008/progress/part2.pdf
    \bibitem{cia}https://www.cia.gov/library/publications/the-world-factbook/rankorder/2155rank.html
\end{thebibliography}
\end{document}
